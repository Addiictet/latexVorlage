%%%%%%%%%%%%%%%%%%%%%%%%%%%%%%%%%%%%%%%%%%%%%%%%%%%%%%%%%%%%%%%%%%%%%%%%%%%%%%%
%                                   Einstellungen
%
% Hier können alle relevanten Einstellungen für diese Arbeit gesetzt werden.
% Dazu gehören Angaben u.a. über den Autor sowie Formatierungen.
%
%
%%%%%%%%%%%%%%%%%%%%%%%%%%%%%%%%%%%%%%%%%%%%%%%%%%%%%%%%%%%%%%%%%%%%%%%%%%%%%%%


%%%%%%%%%%%%%%%%%%%%%%%%%%%%%%%%%%%% Sprache %%%%%%%%%%%%%%%%%%%%%%%%%%%%%%%%%%%
%% Aktuell sind Deutsch und Englisch unterstützt.
%% Es werden nicht nur alle vom Dokument erzeugten Texte in
%% der entsprechenden Sprache angezeigt, sondern auch weitere
%% Aspekte angepasst, wie z.B. die Anführungszeichen und
%% Datumsformate.
\setzesprache{de} % oder en
%%%%%%%%%%%%%%%%%%%%%%%%%%%%%%%%%%%%%%%%%%%%%%%%%%%%%%%%%%%%%%%%%%%%%%%%%%%%%%%%

%%%%%%%%%%%%%%%%%%%%%%%%%%%%%%%%%%% Angaben  %%%%%%%%%%%%%%%%%%%%%%%%%%%%%%%%%%%
%% Die meisten der folgenden Daten werden auf dem
%% Deckblatt angezeigt, einige auch im weiteren Verlauf
%% des Dokuments.
\setzemartrikelnr{1234510}
\setzekurs{ABC2008DE}
\setzetitel{In der Regel haben wir einen zweizeiligen Bachelorthesistitel}
\setzedatumAbgabe{August 2011}
\setzefirma{Firma GmbH}
\setzefirmenort{Firmenort}
\setzeabgabeort{Abgabeort}
\setzeabschluss{Bachelor of Engineering}
\setzestudiengang{Studienganges Vorderasiatische Archäologie}
\setzedhbw{Karlsruhe}
\setzebetreuer{Dipl.-Ing. (FH) Peter Pan}
\setzegutachter{Dr.\ Silvana Koch-Mehrin}
\setzezeitraum{12 Wochen}
\setzearbeit{Bachelorarbeit}
\setzeautor{Vorname Nachname}
%%%%%%%%%%%%%%%%%%%%%%%%%%%%%%%%%%%%%%%%%%%%%%%%%%%%%%%%%%%%%%%%%%%%%%%%%%%%%%%%

%%%%%%%%%%%%%%%%%%%%%%%%%%%% Literaturverzeichnis %%%%%%%%%%%%%%%%%%%%%%%%%%%%%%
%% Bei Fehlern während der Verarbeitung bitte in ads/header.tex bei der
%% Einbindung des Pakets biblatex (ungefähr ab Zeile 110,
%% einmal für jede Sprache), biber in bibtex ändern.
\newcommand{\ladeliteratur}{%
\addbibresource{ArbeitBib.bib}
%\addbibresource{weitereDatei.bib}
}
%%%%%%%%%%%%%%%%%%%%%%%%%%%%%%%%%%%%%%%%%%%%%%%%%%%%%%%%%%%%%%%%%%%%%%%%%%%%%%%%

%%%%%%%%%%%%%%%%%%%%%%%%%%%%%%%%% Layout %%%%%%%%%%%%%%%%%%%%%%%%%%%%%%%%%%%%%%%
%% Verschiedene Schriftarten
\setzeschriftart{palatino} % oder goudysans, lmodern, libertine
%% Paket um Textteile drehen zu können
%\usepackage{rotating}
%% Paket um Seite im Querformat anzuzeigen 
%\usepackage{lscape}
%% Spaltenabstand
\setzespaltenabstand{10pt}
%%Zeilenabstand
\setzezeilenabstand{1.5}
%%%%%%%%%%%%%%%%%%%%%%%%%%%%%%%%%%%%%%%%%%%%%%%%%%%%%%%%%%%%%%%%%%%%%%%%%%%%%%%%

%%%%%%%%%%%%%%%%%%%%%%%%%%%%% Verschiedenes  %%%%%%%%%%%%%%%%%%%%%%%%%%%%%%%%%%%
%% Farben
\newcommand{\ladefarben}{%
  \definecolor{LinkColor}{rgb}{0,0,0.2}
  \definecolor{ListingBackground}{rgb}{0.92,0.92,0.92}
}
%% Mathematikpakete benutzen (Pakete aktivieren)
%\usepackage{amsmath}
%\usepackage{amssymb}
%% Programmiersprachen Highlighting (Listings)
\newcommand{\listingsettings}{%
  \lstset{%
    language=Java,		 	     % Standardsprache des Quellcodes
    %numbers=left,           % Zeilennummern links
    stepnumber=1,            % Jede Zeile nummerieren.
    numbersep=5pt,           % 5pt Abstand zum Quellcode
    numberstyle=\tiny,       % Zeichengrösse 'tiny' für die Nummern.
    breaklines=true,         % Zeilen umbrechen wenn notwendig.
    breakautoindent=true,    % Nach dem Zeilenumbruch Zeile einrücken.
    postbreak=\space,        % Bei Leerzeichen umbrechen.
    tabsize=2,               % Tabulatorgrösse 2
    basicstyle=\ttfamily\footnotesize, % Nichtproportionale Schrift, klein für den Quellcode
    showspaces=false,        % Leerzeichen nicht anzeigen.
    showstringspaces=false,  % Leerzeichen auch in Strings ('') nicht anzeigen.
    extendedchars=true,      % Alle Zeichen vom Latin1 Zeichensatz anzeigen.
    captionpos=b,            % sets the caption-position to bottom
    backgroundcolor=\color{ListingBackground} % Hintergrundfarbe des Quellcodes setzen.
  }
}
%%%%%%%%%%%%%%%%%%%%%%%%%%%%%%%%%%%%%%%%%%%%%%%%%%%%%%%%%%%%%%%%%%%%%%%%%%%%%%%%

%%%%%%%%%%%%%%%%%%%%%%%%%%%%%%%% Eigenes %%%%%%%%%%%%%%%%%%%%%%%%%%%%%%%%%%%%%%%
%% Hier können Ergänzungen zur Präambel vorgenommen werden (eigene Pakete, Einstellungen)

