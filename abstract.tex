\thispagestyle{empty}

\renewcommand{\abstractname}{Zusammenfassung}
\begin{abstract}
Ein Abstract ist eine prägnante Inhaltsangabe, ein Abriss ohne
Interpretation und Wertung einer wissenschaftlichen Arbeit. In DIN
1426 wird das (oder auch der) Abstract als Kurzreferat zur
Inhaltsangabe beschrieben.

\section*{Allgemeine Merkmale}
\begin{description}
\item[Objektivität] soll sich jeder persönlichen Wertung enthalten
\item[Kürze] soll so kurz wie möglich sein
\end{description}

\section*{Verwendung}
Üblicherweise müssen wissenschaftliche Artikel einen Abstract
enthalten, typischerweise von 100-150 Wörtern, ohne Bilder und
Literaturzitate und in einem Absatz.

Quelle \url{http://de.wikipedia.org/wiki/Abstract} Abgerufen 07.07.2011
\end{abstract}


\renewcommand{\abstractname}{Summary}
\begin{abstract}
An abstract is a brief summary of a research article, thesis, review,
conference proceeding or any in-depth analysis of a particular subject
or discipline, and is often used to help the reader quickly ascertain
the paper's purpose. When used, an abstract always appears at the
beginning of a manuscript, acting as the point-of-entry for any given
scientific paper or patent application. Abstracting and indexing
services for various academic disciplines are aimed at compiling a
body of literature for that particular subject.

The terms précis or synopsis are used in some publications to refer to
the same thing that other publications might call an "abstract". In
management reports, an executive summary usually contains more
information (and often more sensitive information) than the abstract
does.

\section*{Purpose}
Academic literature uses the abstract to succinctly communicate
complex research. An abstract may act as a stand-alone entity instead
of a full paper. As such, an abstract is used by many organizations as
the basis for selecting research that is proposed for presentation in
the form of a poster, platform/oral presentation or workshop
presentation at an academic conference. Most literature database
search engines index only abstracts rather than providing the entire
text of the paper. Full texts of scientific papers must often be
purchased because of copyright and/or publisher fees and therefore the
abstract is a significant selling point for the reprint or electronic
form of the full text.

Quelle: \url{http://en.wikipedia.org/wiki/Abstract_(summary)}

\end{abstract}
