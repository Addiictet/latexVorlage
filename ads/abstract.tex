\pagestyle{empty}

\renewcommand{\abstractname}{Zusammenfassung}
\begin{abstract}
Ein Abstract ist eine prägnante Inhaltsangabe, ein Abriss ohne
Interpretation und Wertung einer wissenschaftlichen Arbeit. In DIN
1426 wird das (oder auch der) Abstract als Kurzreferat zur
Inhaltsangabe beschrieben.

\begin{description}
\item[Objektivität] soll sich jeder persönlichen Wertung enthalten
\item[Kürze] soll so kurz wie möglich sein
\item[Genauigkeit] soll genau die Inhalte und die Meinung der Originalarbeit wiedergeben
\end{description}

Üblicherweise müssen wissenschaftliche Artikel einen Abstract
enthalten, typischerweise von 100-150 Wörtern, ohne Bilder und
Literaturzitate und in einem Absatz.

Quelle \url{http://de.wikipedia.org/wiki/Abstract} Abgerufen 07.07.2011
\end{abstract}


\renewcommand{\abstractname}{Summary}
\begin{abstract}
An abstract is a brief summary of a research article, thesis, review,
conference proceeding or any in-depth analysis of a particular subject
or discipline, and is often used to help the reader quickly ascertain
the paper's purpose. When used, an abstract always appears at the
beginning of a manuscript, acting as the point-of-entry for any given
scientific paper or patent application. Abstracting and indexing
services for various academic disciplines are aimed at compiling a
body of literature for that particular subject.

The terms précis or synopsis are used in some publications to refer to
the same thing that other publications might call an "abstract". In
management reports, an executive summary usually contains more
information (and often more sensitive information) than the abstract
does.

Quelle: \url{http://en.wikipedia.org/wiki/Abstract_(summary)}

\end{abstract}
