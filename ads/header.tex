%!TEX root = ../dokumentation.tex

%
% Nahezu alle Einstellungen koennen hier getaetigt werden
%

\documentclass[%
  pdftex,
  oneside,		% Einseitiger Druck.
  12pt,			% Schriftgroesse
  parskip=half,	% Halbe Zeile Abstand zwischen Absätzen.
  headsepline,	% Linie nach Kopfzeile.
  footsepline,	% Linie vor Fusszeile.
  abstracton,	    % Abstract Überschriften
  ngerman,		% Translator
  listof=totoc,
  toc=bibliography,
]{scrreprt}

% für String-Vergleiche
\usepackage{xstring}

% Zeichencodierung
\usepackage[utf8]{inputenc}
\usepackage[T1]{fontenc}

% für Einstellungen
\newcommand{\einstellung}[1]{%
  \expandafter\newcommand\csname #1\endcsname{}
  \expandafter\newcommand\csname setze#1\endcsname[1]{\expandafter\renewcommand\csname#1\endcsname{##1}}
}
\newcommand{\langstr}[1]{\einstellung{lang#1}}

% verfügbare Einstellungen
\einstellung{martrikelnr}
\einstellung{titel}
\einstellung{kurs}
\einstellung{datumAbgabe}
\einstellung{firma}
\einstellung{firmenort}
\einstellung{abgabeort}
\einstellung{abschluss}
\einstellung{studiengang}
\einstellung{dhbw}
\einstellung{betreuer}
\einstellung{gutachter}
\einstellung{zeitraum}
\einstellung{arbeit}
\einstellung{autor}
\einstellung{sprache}
\einstellung{schriftart}
\einstellung{spaltenabstand}
\einstellung{zeilenabstand}
\einstellung{zitierstil}

% lese Einstellungen
%%%%%%%%%%%%%%%%%%%%%%%%%%%%%%%%%%%%%%%%%%%%%%%%%%%%%%%%%%%%%%%%%%%%%%%%%%%%%%%
%                                   Einstellungen
%
% Hier können alle relevanten Einstellungen für diese Arbeit gesetzt werden.
% Dazu gehören Angaben u.a. über den Autor sowie Formatierungen.
%
%
%%%%%%%%%%%%%%%%%%%%%%%%%%%%%%%%%%%%%%%%%%%%%%%%%%%%%%%%%%%%%%%%%%%%%%%%%%%%%%%


%%%%%%%%%%%%%%%%%%%%%%%%%%%%%%%%%%%% Sprache %%%%%%%%%%%%%%%%%%%%%%%%%%%%%%%%%%%
%% Aktuell sind Deutsch und Englisch unterstützt.
%% Es werden nicht nur alle vom Dokument erzeugten Texte in
%% der entsprechenden Sprache angezeigt, sondern auch weitere
%% Aspekte angepasst, wie z.B. die Anführungszeichen und
%% Datumsformate.
\setzesprache{de} % oder en
%%%%%%%%%%%%%%%%%%%%%%%%%%%%%%%%%%%%%%%%%%%%%%%%%%%%%%%%%%%%%%%%%%%%%%%%%%%%%%%%

%%%%%%%%%%%%%%%%%%%%%%%%%%%%%%%%%%% Angaben  %%%%%%%%%%%%%%%%%%%%%%%%%%%%%%%%%%%
%% Die meisten der folgenden Daten werden auf dem
%% Deckblatt angezeigt, einige auch im weiteren Verlauf
%% des Dokuments.
\setzemartrikelnr{1234510}
\setzekurs{ABC2008DE}
\setzetitel{In der Regel haben wir einen zweizeiligen Bachelorthesistitel}
\setzedatumAbgabe{August 2011}
\setzefirma{Firma GmbH}
\setzefirmenort{Firmenort}
\setzeabgabeort{Abgabeort}
\setzeabschluss{Bachelor of Engineering}
\setzestudiengang{Vorderasiatische Archäologie}
\setzedhbw{Karlsruhe}
\setzebetreuer{Dipl.-Ing. (FH) Peter Pan}
\setzegutachter{Dr.\ Silvana Koch-Mehrin}
\setzezeitraum{12 Wochen}
\setzearbeit{Bachelorarbeit}
\setzeautor{Vorname Nachname}
%%%%%%%%%%%%%%%%%%%%%%%%%%%%%%%%%%%%%%%%%%%%%%%%%%%%%%%%%%%%%%%%%%%%%%%%%%%%%%%%

%%%%%%%%%%%%%%%%%%%%%%%%%%%% Literaturverzeichnis %%%%%%%%%%%%%%%%%%%%%%%%%%%%%%
%% Bei Fehlern während der Verarbeitung bitte in ads/header.tex bei der
%% Einbindung des Pakets biblatex (ungefähr ab Zeile 110,
%% einmal für jede Sprache), biber in bibtex ändern.
\newcommand{\ladeliteratur}{%
\addbibresource{bibliographie.bib}
%\addbibresource{weitereDatei.bib}
}
%%%%%%%%%%%%%%%%%%%%%%%%%%%%%%%%%%%%%%%%%%%%%%%%%%%%%%%%%%%%%%%%%%%%%%%%%%%%%%%%

%%%%%%%%%%%%%%%%%%%%%%%%%%%%%%%%% Layout %%%%%%%%%%%%%%%%%%%%%%%%%%%%%%%%%%%%%%%
%% Verschiedene Schriftarten
\setzeschriftart{palatino} % oder goudysans, lmodern, libertine
%% Paket um Textteile drehen zu können
%\usepackage{rotating}
%% Paket um Seite im Querformat anzuzeigen
%\usepackage{lscape}
%% Spaltenabstand
\setzespaltenabstand{10pt}
%%Zeilenabstand
\setzezeilenabstand{1.5}
%%%%%%%%%%%%%%%%%%%%%%%%%%%%%%%%%%%%%%%%%%%%%%%%%%%%%%%%%%%%%%%%%%%%%%%%%%%%%%%%

%%%%%%%%%%%%%%%%%%%%%%%%%%%%% Verschiedenes  %%%%%%%%%%%%%%%%%%%%%%%%%%%%%%%%%%%
%% Farben (Angabe in HTML-Notation mit großen Buchstaben)
\newcommand{\ladefarben}{%
	\definecolor{LinkColor}{HTML}{00007A}
	\definecolor{ListingBackground}{HTML}{FCF7DE}
}
%% Mathematikpakete benutzen (Pakete aktivieren)
%\usepackage{amsmath}
%\usepackage{amssymb}

%% Programmiersprachen Highlighting (Listings)
\newcommand{\listingsettings}{%
	\lstset{%
		language=Java,			% Standardsprache des Quellcodes
		%numbers=left,			% Zeilennummern links
		stepnumber=1,			% Jede Zeile nummerieren.
		numbersep=5pt,			% 5pt Abstand zum Quellcode
		numberstyle=\tiny,		% Zeichengrösse 'tiny' für die Nummern.
		breaklines=true,		% Zeilen umbrechen wenn notwendig.
		breakautoindent=true,	% Nach dem Zeilenumbruch Zeile einrücken.
		postbreak=\space,		% Bei Leerzeichen umbrechen.
		tabsize=2,				% Tabulatorgrösse 2
		basicstyle=\ttfamily\footnotesize, % Nichtproportionale Schrift, klein für den Quellcode
		showspaces=false,		% Leerzeichen nicht anzeigen.
		showstringspaces=false,	% Leerzeichen auch in Strings ('') nicht anzeigen.
		extendedchars=true,		% Alle Zeichen vom Latin1 Zeichensatz anzeigen.
		captionpos=b,			% sets the caption-position to bottom
		backgroundcolor=\color{ListingBackground}, % Hintergrundfarbe des Quellcodes setzen.
		xleftmargin=0pt,		% Rand links
		xrightmargin=0pt,		% Rand rechts
		frame=single,			% Rahmen an
		frameround=ffff,
		rulecolor=\color{darkgray},	% Rahmenfarbe
		fillcolor=\color{ListingBackground}
	}
}
%%%%%%%%%%%%%%%%%%%%%%%%%%%%%%%%%%%%%%%%%%%%%%%%%%%%%%%%%%%%%%%%%%%%%%%%%%%%%%%%

%%%%%%%%%%%%%%%%%%%%%%%%%%%%%%%% Eigenes %%%%%%%%%%%%%%%%%%%%%%%%%%%%%%%%%%%%%%%
%% Hier können Ergänzungen zur Präambel vorgenommen werden (eigene Pakete, Einstellungen)



% Sprache einlesen
\newcommand{\iflang}[2]{%
  \IfStrEq{\sprache}{#1}{#2}{}
}

\langstr{abkverz}
\langstr{anhang}
\langstr{glossar}
\langstr{deckblattabschlusshinleitung}
\langstr{artikelstudiengang}
\langstr{studiengang}
\langstr{anderdh}
\langstr{von}
\langstr{dbbearbeitungszeit}
\langstr{dbmatriknr}
\langstr{dbkurs}
\langstr{dbfirma}
\langstr{dbbetreuer}
\langstr{dbgutachter}
\langstr{sperrvermerk}
\langstr{erklaerung}
\langstr{abstract}
\langstr{listingname}
\langstr{listlistingname}
\langstr{listingautorefname}

\input{lang/\sprache}

%Seitengroesse
\usepackage{fullpage}

%Zeilenumbruch und mehr
\usepackage[activate]{microtype}

% Zeilenabstand
\usepackage[onehalfspacing]{setspace}

% Index-Erstellung
\usepackage{makeidx}

% Lokalisierung (neue deutsche Rechtschreibung)
\iflang{de}{\usepackage[ngerman]{babel}}
\iflang{en}{\usepackage[english]{babel}}

% Anführungszeichen 
\usepackage[autostyle=true,german=quotes]{csquotes}

% Spezielle Tabellenform fuer Deckblatt
\usepackage{longtable}
\setlength{\tabcolsep}{10pt} %Abstand zwischen Spalten
\renewcommand{\arraystretch}{1.5} %Zeilenabstand

% Grafiken
\usepackage{graphicx}

% Mathematische Textsaetze
%\usepackage{amsmath}
%\usepackage{amssymb}

% Pakete um Textteile drehen zu können, oder eine Seite Querformat anzeigen kann.
%\usepackage{rotating}
%\usepackage{lscape}

% Farben
\usepackage{color}
\definecolor{LinkColor}{rgb}{0,0,0.2}
\definecolor{ListingBackground}{rgb}{0.92,0.92,0.92}

% Titel, Autor und Datum
\title{\titel}
\author{\autor}
\date{\datum}

% PDF Einstellungen
\usepackage[%
	pdftitle={\titel},
	pdfauthor={\autor},
	pdfsubject={\arbeit},
	pdfcreator={pdflatex, LaTeX with KOMA-Script},
	pdfpagemode=UseOutlines, % Beim Oeffnen Inhaltsverzeichnis anzeigen
	pdfdisplaydoctitle=true, % Dokumenttitel statt Dateiname anzeigen.
	pdflang=de % Sprache des Dokuments.
]{hyperref} 

% (Farb-)einstellungen für die Links im PDF
\hypersetup{%
	colorlinks=true, % Aktivieren von farbigen Links im Dokument
	linkcolor=LinkColor, % Farbe festlegen
	citecolor=LinkColor,
	filecolor=LinkColor,
	menucolor=LinkColor,
	urlcolor=LinkColor,
	bookmarksnumbered=true % Überschriftsnummerierung im PDF Inhalt anzeigen.
}

% Literaturverweise (mit deutschem und)
\iflang{de}{%
\usepackage[
	    backend=biber,      % empfohlen. Falls biber nicht installierbar: bibtex
	    bibwarn=true,
	    bibencoding=utf8, 	% wenn .bib in utf8, sonst ascii
	    sortlocale=de_DE,
	    style=alphabetic 	%Zitierstil. Siehe http://ctan.mirrorcatalogs.com/macros/latex/contrib/biblatex/doc/biblatex.pdf 
]{biblatex}
}
\iflang{en}{%
\usepackage[
	    backend=biber,      % empfohlen. Falls biber nicht installierbar: bibtex
	    bibwarn=true,
	    bibencoding=utf8, 	% wenn .bib in utf8, sonst ascii
	    sortlocale=en_US,
	    style=alphabetic 	%Zitierstil. Siehe http://ctan.mirrorcatalogs.com/macros/latex/contrib/biblatex/doc/biblatex.pdf 
]{biblatex}

}
\addbibresource{ArbeitBib.bib}

% Verschiedene Schriftarten
%\usepackage{goudysans}
%\usepackage{lmodern}
%\usepackage{libertine}
\usepackage{palatino} 

% Hurenkinder und Schusterjungen verhindern
% http://projekte.dante.de/DanteFAQ/Silbentrennung
\clubpenalty=10000
\widowpenalty=10000
\displaywidowpenalty=10000

% Quellcode
\usepackage{listings}
\lstloadlanguages{Python}
\lstset{%
	language=Python,		 	 % Sprache des Quellcodes
	%numbers=left,           % Zelennummern links
	stepnumber=1,            % Jede Zeile nummerieren.
	numbersep=5pt,           % 5pt Abstand zum Quellcode
	numberstyle=\tiny,       % Zeichengrösse 'tiny' für die Nummern.
	breaklines=true,         % Zeilen umbrechen wenn notwendig.
	breakautoindent=true,    % Nach dem Zeilenumbruch Zeile einrücken.
	postbreak=\space,        % Bei Leerzeichen umbrechen.
	tabsize=2,               % Tabulatorgrösse 2
	basicstyle=\ttfamily\footnotesize, % Nichtproportionale Schrift, klein für den Quellcode
	showspaces=false,        % Leerzeichen nicht anzeigen.
	showstringspaces=false,  % Leerzeichen auch in Strings ('') nicht anzeigen.
	extendedchars=true,      % Alle Zeichen vom Latin1 Zeichensatz anzeigen.
	captionpos=b,            % sets the caption-position to bottom
	backgroundcolor=\color{ListingBackground} % Hintergrundfarbe des Quellcodes setzen.
}

% Glossar
\usepackage[
	nonumberlist, %keine Seitenzahlen anzeigen
	%acronym,      %ein Abkürzungsverzeichnis erstellen
	%section,     %im Inhaltsverzeichnis auf section-Ebene erscheinen
	toc,          %Einträge im Inhaltsverzeichnis
]{glossaries}

%Akronyme
\usepackage[printonlyused,footnote]{acronym}

% Fussnoten
\usepackage[perpage, hang, multiple, stable]{footmisc}

%Bildpfad
\graphicspath{{images/}}

%nur ein latex-Durchlauf für die Aktualisierung von Verzeichnissen nötig
\usepackage{bookmark}

%Gleitumgebungen (Bilder, Tabellen, usw\ldots) lassen sich mit H an genau der
% definierten Stelle platzieren
\usepackage{float}

% für die vertikale Platzierung von Text in Tabellen
\usepackage{array}

% für die Darstellung des Euro-Symbols
\usepackage[right]{eurosym}

% für textumflossene Grafiken
\usepackage{wrapfig}

% eine Kommentarumgebung "k" (Handhabe mit \begin{k}<Kommentartext>\end{k},
% Kommentare werden rot gedruckt). Wird \% vor excludecomment{k} entfernt,
% werden keine Kommentare mehr gedruckt.
\usepackage{comment}
\specialcomment{k}{\begingroup\color{red}}{\endgroup}
%\excludecomment{k}

% für automatische Kapiteldateieinbindung
\usepackage{pgffor}

